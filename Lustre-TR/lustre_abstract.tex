\begin{abstract}

Lustre was initiated and funded, almost a decade ago, by the U.S. Department
of Energy Office of Science and National Nuclear Security Administration
laboratories to address the need for an open source, highly scalable,
high-performance parallel filesystem on then-present and future
supercomputing platforms. Throughout the last decade, it was deployed over
numerous medium- to large-scale supercomputing platforms and clusters, and it
performed and met the expectations of the Lustre user community. 
At the time of this writing, according to the Top500 list, 15 of the
top 30 supercomputers in the world use Lustre filesystem. 

This report aims to present a streamlined overview of how Lustre works
internally at reasonable detail including relevant data structures, APIs,
protocols, and algorithms involved for the Lustre version 1.6 source code
base.  More important, the report attempts to explain how various components
interconnect and function as a system. Portions of the report are based on
discussions with Oak Ridge National Laboratory Lustre Center of Excellence
team members, and portions of it are based on the authors' understanding of
how the code works. We, the authors, bear all responsibility for errors and
omissions in this document. We can only hope the report helps current and
future Lustre users and Lustre code developers as much as it helped us
understanding the Lustre source code and its internal workings.  

\end{abstract}

