\section{Lustre Numbers}

Anything to do with number is of interests here.


\begin{itemize}

\item Stripe Information

\begin{Verbatim}
    # -s 0  : default stripe size, otherwise, multiple of 64KB
    # -i -1 : random first OST is chosen, 0 to N-1 is OK
    # -c -1 : all available OSTs are used; 0 if default number of OSTs  
    lfs setstripe -s 0 -i -1 -c -1
\end{Verbatim}

\item Free space management: weighted allocator is used when any two OSTs are
imbalanced by more than 20\%. Until then, roound robin allocator is used.

\begin{Verbatim}
    lctl conf_param <fsname>-MDT0000.lov.qos_prio_free=90
\end{Verbatim}


\item Client 32-bit checksum is enabled by default:

\begin{Verbatim}
    echo 0 > /proc/fs/lustre/llite/{fsname}/checksum_pages
\end{Verbatim}


\item OST limit right now is 8TB, and it will be 16TB when ldiskfs migrates to
ext4. 

\item MDT device can be up to 8TB in size. With default configuration, you get
one inode for every 4KB, but you can double the inodes if you format it with
one inode every 2KB.

\item Dirty cache limit on a per-OSC basis is tunnable by \url{max_dirty_mb},
default is 3/4 of RAM.

\begin{quote}

\url{get_param} and \url{set_param}: Up to 1.6.5, these two \url{lctl} actions
are supported, and is a preferred way to doing parameter changes than querying
the \url{/proc}. An example Andreas gave is:
\smallskip
\begin{Verbatim}
lctl set_param osc.*.max_dirty_mb = 512
\end{Verbatim}
\smallskip
instead of:
\smallskip
\begin{Verbatim}
for x in /proc/fs/lustre/osc/*/max_dirty_mb; do
    echo 512 > $x
done
\end{Verbatim}

\end{quote}

\item Dirty cache limit per client:
\url{/proc/fs/lustre/llite/*/max_cached_mb}, default is 32MB.


\item Read ahead, default size is 40MB

\item OSS cache (1.8 release), the tunable
\url{/proc/fs/lustre/obdfilter/*/readcache_max_filesize} will limit the
maximum file size that is cached on the OSS so that small files can be cached,
and large files will not wipe out the read cache.



\item To find out what network/IP address is provisioning which OST target,
or MDT target for that matter.

\begin{Verbatim}
lctl get_param osc.${fsname}-${OSTname}*.ost_conn_uuid
lctl get_param mdc.${fsname}-${MDTname}*.mds_conn_uuid
\end{Verbatim}

There will be a \url{lct list_param} command in upcoming release to get all
valid parameters you can play.


\item The number of client currently mounted? Check
\url{/proc/fs/lustre/mds/num_exports}. Note that each server has an internal
reference to the device, so this file is number of clients plus 1.


\item The largest memory ever used by the Lustre code itself? check
\url{lustre.memused_max} parameter.


\end{itemize}





